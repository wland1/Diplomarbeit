\chapter{Versuchsvorbereitung}\label{ch:vorbereitung}

Die Verfügbarkeit von Literaturdaten zu Einflüssen auf die Haftung von Kunststoff – Magnesiumblech ist gering.
Um die Einflüsse verschiedener Vorbehandlungsmethoden auf die Haftung des Verbundes einschätzen zu können, werden Scherzugversuche durchgeführt.
Die Durchführung der Versuche orientiert sich an DIN-Normen, welche die Geometrie und andere Eigenschaften der Proben festlegen.
Die Anwendung des untersuchten Verbundes beinhaltet Biegeoperationen, welche mithilfe von Biegeversuchen untersucht werden sollen.
Dabei wird eine anwendungsbezogene Geometrie genutzt, welche in das konstruierte Werkzeug übernommen wird.

Um die Anzahl der gefertigten Proben zu reduzieren, wurde eine zweistufige Untersuchung durchgeführt.
Diese besteht im ersten Schritt aus einer Untersuchung der Haftung durch Scherzugversuche.
Anhand dieser Untersuchung wurde eine Vorbehandlungskombination ausgewählt, welche sehr gute Haftungseigenschaften aufweist, und in Biegeversuchen weiter untersucht.
Durch Biegeversuche werden die Möglichkeiten der Biegeumformung des Verbundes untersucht sowie die Arten des Versagens der Grenzschicht.

\section{Scherzugversuche}\label{sec:scherzug}

\subsection{Probengeometrie}\label{subsec:probengeometrie}

\subsection{Probenherstellung}\label{subsec:probenherstellung}

\section{Biegeversuche}\label{sec:Biegen}

\subsection{Probengeometrie}\label{subsec:probengeometrie2}

\subsection{Probenherstellung}\label{subsec:probenherstellung2}

\section{Versuchsplanung}\label{sec:Planung}