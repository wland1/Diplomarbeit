\chapter{Einleitung}\label{ch:Einleit}

In den vergangenen Jahrzehnten wurden die Bemühungen verstärkt, monolithische und hybride Leichtbauwerkstoffe weiterzuentwickeln und anzuwenden.
Als besonderer Treiber dieser Entwicklung kann die Luftfahrtbranche angesehen werden, die unter anderem in den 70er und 80er Jahren des zwanzigsten Jahrhunderts große Fortschritte erzielen konnte.
So konnten unter Anwendung von Faserverbundwerkstoffen hochfeste Schichtwerkstoffe entwickelt werden, die im Flugzeugbau seit vielen Jahren Anwendung finden.

Sogenannte Fiber Metal Laminates (FML, deutsch: Faser-Metall-Laminate) zeichnen sich dabei unter anderem durch hohe spezifische Festigkeit, hohe Steifigkeit und günstiges Versagensverhalten aus.
Durch die Verwendung von symmetrischen oder asymmetrischen Verbundwerkstoffen in verschiedenen Konfigurationen können auch andere Eigenschaften wie Anisotropie, Hochtemperaturfestigkeit oder Energieaufnahme beim Versagen beeinflusst werden.
Eine weitere Möglichkeit zur Beeinflussung der Eigenschaften ist die Verwendung verschiedener Metall-, Matrix- und Faserwerkstoffe.
Die häufig verwendete Werkstoffkombinationen aus mehreren Lagen hochfesten Aluminiumblechen mit einem epoxidharzgebundenen Glas- oder Kohlenstofffasergewebe (Glare oder Arall) zwischen den Metalllagen finden in modernen Flugzeugen Verwendung, beispielsweise als Rumpfbauteile.

FML erweisen sich hinsichtlich ihrer Eigenschaften als vorteilhaft gegenüber anderen Werkstoffen, sind aber aufgrund der relativ langsamen und aufwendigen Produktion bisher nicht in die breite Anwendung übernommen worden. 
Seit einigen Jahren werden deshalb andere Werkstoffkombinationen untersucht.
Dazu zählt unter anderem die Verwendung von anderen Aluminiumlegierungen sowie Stahl, Titan oder auch Magnesium als Metallschicht.
Die Verwendung von Magnesium als leichtestem Konstruktionsmetall erscheint dabei aussichtsreich.
Werkstoffseitig wird der Einsatz von thermoplastischen Polymeren statt dem bisher dominierendem Epoxidharz untersucht. 

Mit diesen Entwicklungen soll das Anwendungsspektrum erweitert werden.
Durch die Verwendung von thermoplastischen Matrixwerkstoffen soll die Herstellung vereinfacht und somit für eine größere Produktpalette geöffnet werden.
Eine aktuelle Entwicklung ist die Übertragung herkömmlicher Umformprozesse auf FML, was zu deutlich sinkenden Produktionskosten führen soll.

Diese Arbeit soll die Möglichkeit der Herstellung und Nutzung eines Verbundwerkstoffes aus der Magnesiumlegierung ZAX210 und einem Polyamidtapes mit Glasfaserverstärkung untersuchen.
Dabei sollen unterschiedliche Möglichkeiten der Oberflächenvorbehandlung betrachtet, ausgewählt und durchgeführt werden.
Verschiedene Versuche wie Scherzug-, Biege- oder Zugversuche werden einbezogen, um den Einfluss der Oberflächenvorbehandlung auf die interlaminare Haftung im Verbund einschätzen zu können.
Die optische Untersuchung der Oberflächenzustände nach der jeweiligen Vorbehandlung sowie der Grenzschichtzustände werden in der Arbeit durchgeführt.

Als Ergebnis soll diese Arbeit eine Empfehlung erarbeiten, auf deren Basis ein Prozess zur Herstellung des genannten Verbundes im Prototypenmaßstab erstellt werden kann.
Deshalb sollen die Möglichkeiten anhand der Produktivität und Übertragbarkeit in einen größeren Maßstab eingeschätzt werden.

%Das Aufbringen der Polymermatrix kann dabei in fester oder schmelzflüssiger Form erfolgen. Ein nachträgliches Aufschmelzen für eventuelle Umformprozesse ist mit einer thermoplastischen Matrix möglich. 

%Die Herstellung von zwei- oder dreidimensionalen Strukturen kann durch den Einsatz von Biege- und Tiefziehprozessen deutlich vereinfacht und die Anwendungspalette erweitert werden. Aktuelle Untersuchungen zeigen die Machbarkeit von dreidimensionalen Tiefziehteilen mit geringem Fertigungsaufwand durch kombinierte Füge- und Umformschritte des Verbundes.

%Ein weiterer Ansatz zur Vereinfachung der Fertigung ist ein kontinuierlicher Aufrollprozess, welcher in einem möglichen Walzprofilierprozess integriert werden kann. Dabei wird ein Laminatwerkstoff, bestehend aus der Matrix und einer Faserverstärkung, aufgeschmolzen und in einem kontinuierlichen Walzprozess mit dem entsprechenden Grundwerkstoff verpresst. 
%Neben Parametern wie Temperaturführung im Fügeprozess, Geschwindigkeit, Presskraft und erforderliche Dicke des Kunststofflaminats ist der Oberflächenzustand des Metalls entscheidend für die resultierende Festigkeit des Verbundes. 
%
%In dieser Arbeit soll die Herstellung eines Verbundes aus einem Blech der Magnesiumlegierung ZAX210 und einem Kuststofftape aus einem Polyamid 6 mit Glasfaserverstärkung untersucht werden. Um eine ausreichend hohe Haftfestigkeit des Verbundes einstellen zu können, sollen verschiedene Möglichkeiten der Oberflächenvorbehandlung überprüft und auf dieser Basis Empfehlungen für eine Umsetzung in einem Pilotprozess erarbeitet werden.
