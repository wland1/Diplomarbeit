\chapter{Zielstellung}\label{ch:ziel}

%\section{Auswahl Werkstoffe und Verfahren}\label{sec:Auswahl}

Verbunde aus Metallblechen und faserverstärkten Kunststoffen werden weiter untersucht und verbreitet eingesetzt.
Ein Fokus der Entwicklung ist die Verwendung von alternativen Leichtbauwerkstoffen wie Magnesium.
Dieses zeichnet sich durch hohe spezifische Festigkeiten aus und kann zu signifikanten Gewichtseinsparungen führen.
Weiterhin sind die guten Dämpfungseigenschaften von Magnesium erwähnenswert.
Magnesium wird in dieser Untersuchung als Blechwerkstoff verwendet.
Da die Umformung des Blechwerkstoffes mitgedacht werden soll, wird die Magnesiumlegierung ZAX210 verwendet, die eine verbesserte Umformbarkeit bei guten mechanischen Eigenschaften ermöglicht.

Aktuelle Untersuchungen widmen sich der Weiterverarbeitung von FML in der automatisierten Fertigung durch Umformen.
Als vielversprechender Ansatz hat sich die Verwendung von faserverstärkten Polymeren herausgestellt, da durch erneutes Erwärmen und Aufschmelzen eine Formgebung von intakten Bauteilen möglich ist.
In dieser Untersuchung wird als Matrixwerkstoff PA6 verwendet, welches als Band mit einer unidirektionalen Endlosfaserverstärkung aus Glasfasern versehen ist.

Tabelle mech Eigenschaften ZAX210 und PA6

Ein Ziel der vorliegenden Untersuchung ist die Erarbeitung von Möglichkeiten der Oberflächenvorbehandlung, mit denen die Haftfestigkeit von Metall-Kunststoff-Verbindungen erhöht werden kann.
Daraus sollen mehrere Möglichkeiten ermittelt werden, die mit vertretbarem Aufwand im Labormaßstab umgesetzt werden können.
Die entsprechenden Vorbehandlungen sollen durch quantitative und qualitative Verfahren untersucht werden.

Um die Haftfestigkeiten des Verbundes bei unterschiedlichen Vorbehandlungen zu validieren, sollen passende Versuchsaufbauten erstellt werden.
Die entsprechenden Proben sollen im Labormaßstab gefertigt werden.
Es sollen Versuche durchgeführt werden, um verschiedene Versagensarten zu untersuchen, unter anderem Biege- und Scherzugbelastung.

Das Versagensverhalten der Proben in unterschiedlichen Vorbehandlungszuständen soll untersucht und beschrieben.
Entsprechende Handlungsempfehlungen zur Verbesserung der Haftfestigkeit des Verbundes sollen aus den Versuchen abgeleitet werden.
Dabei soll eine Bewertung der möglichen Technologien hinsichtlich des Nutzens und der Anwendbarkeit im großtechnischen Maßstab erarbeitet werden.